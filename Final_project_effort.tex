\documentclass[english]{article}
\usepackage[T1]{fontenc}
\usepackage[latin9]{inputenc}
\usepackage{babel}

\begin{document}

\title{CSci 4511W Final Project {}``Effort Paper''}


\author{Sean Bowman}

\maketitle
As I am the sole member of this project group, I of course did all
work presented in the real paper. I implemented everything described
from scratch in Python, using no third party libraries or other peoples'
code. A description of each file in the system follows:
\begin{itemize}
\item LogicTokenizer.py

\begin{itemize}
\item Tokenizes an input string, converting a set of characters to a more
computer friendly form. Example: {}``a => b'' becomes {[}Token('var',
'a'), Token('implies'), Token('var', 'b'){]}
\end{itemize}
\item LogicParser.py

\begin{itemize}
\item Parses an input string with the help of LogicTokenizer, converting
any input expression into an Expression class representing its meaning.
\end{itemize}
\item LogicSimplifier.py

\begin{itemize}
\item Several methods to simplify a logical statement, culminating in to\_cnf()
which converts any input Expression into conjunctive normal form.
\end{itemize}
\item KnowledgeBase.py

\begin{itemize}
\item The actual resolution system, taking input from users and, through
the help of LogicSimplifier and LogicParser, acts as a functioning
knowledge based as described in the main paper.
\end{itemize}
\item testlogic.py

\begin{itemize}
\item My little test bed for the knowledge base. Reads several test cases
from {}``tests.txt'' (the syntax of that text file is pretty intuitive),
timing how long it takes to prove the desired result and ensuring
the correctness of that result.
\end{itemize}
\end{itemize}
I believe I presented the work done clearly in the real paper, however
I will go over several bullet points here:
\begin{itemize}
\item Implementing the parsing system

\begin{itemize}
\item I initially tried to implement unnecessarily complex parsing algorithms,
i.e. a recursive descent parser, the CYK algorithm, etc. I eventually
realized that I can make a simple infix parser work with a few modifications.
\item To do this I had to learn about regular expressions, parsing of languages,
etc. Took me a while to do actually.
\end{itemize}
\item Logic simplification algorithms

\begin{itemize}
\item This was the brunt of my work, I believe. Even once I figured out
how to conveniently express logical sentences in Python (my final
solution was a single {}``Expression'' class, whose actual meaning
varies based on its contents), it was quite a bit of work translating
logical entailment theorems into code. 
\item What complicated things is that everything has to be done recursively:
e.g. once implications are eliminated at one level, you have to recursively
call the function on each of the sub-expressions as well. 
\item The most difficult rule to implement was the distribute\_or\_over\_and
function. Handling a case such as $a\lor\left(b\land c\right)$ was
very simple, however the complexity needed to increase by quite a
bit to handle cases like $\left(a\land b\right)\lor\left(c\land d\right)$
appropriately. The final result handles all cases well, no matter
how large. $\left(a\land b\land c\land d\right)\lor\left(e\land f\land g\land h\right)\lor\left(i\land j\land k\land l\right)$
is no match for my algorithm (the result is a beast containing 64
clauses).
\end{itemize}
\item Resolution System

\begin{itemize}
\item This was a bit tricky to get to work correctly, though once everything
was in CNF form from the logic simplification functions this wasn't
terrible.
\item Implementing the resolution algorithms did cause me to shift between
representations before I settled on what was most convenient, though.
Initially I tried representing clauses as their actual logical expressions,
e.g. (a or b or c). After trying to work with that for a bit, I realized
it was needless and cumbersome, and decided to convert them to sets
before resolving, e.g. $\left\{ a,b,c\right\} $. 
\end{itemize}
\end{itemize}
Nearly all of this project was completely new to me (parsers, regular
expressions, logic programming of any sort, etc.), and I learned a
lot in the process of creating this system. 
\end{document}
